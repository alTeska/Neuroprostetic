\documentclass{scrartcl}			% defines the kind of document you want to produce

% Include different packages:
\usepackage[utf8]{inputenc}
\usepackage[T1]{fontenc}
\usepackage{lmodern}
\usepackage[english]{babel}
\usepackage{amsmath}
\usepackage{graphicx}           	% include graphics
\usepackage{caption}	
\usepackage{subcaption}	 
\usepackage{hyperref}

\title{Neuroprothetik Exercise \\ Template}
\author{Korbinian Steger}
\date{28. Oktober 2016}


\begin{document} 					% Document begins here

\maketitle

\section{Heading section one}		% start a new section 

This is the first section. Here you can start writing your solution to the exercise.

\subsection{Heading subsection}		% start a new subsection 

You can use subsections to devide bigger tasks.

\subsection{Equations}

\LaTeX{} is really good for working with equations. Here are 2 very popular equations by Einstein:

\begin{align}						% the character behind "&" will be aligned
E &= mc^2                                  \\%new line
m &= \frac{m_0}{\sqrt{1-\frac{v^2}{c^2}}}
\end{align}

\subsection{Figures}
\label{subsec_fig} %choose a label, see subsection references

You can of course also import figures of various formats including .jpg and vector-based figures. Figure \ref{fig:figure} shows one big example of using \LaTeX{}.

\begin{figure*}[hbpt!]					%start figure-environment
	\centering
	\includegraphics[]{figures/figure.jpg}
	\captionsetup{width=\linewidth}  %choose the with of the caption
	\caption{Here goes the caption for the figure.}
	\label{fig:figure} %choose a label, see subsection references
\end{figure*}

\subsection{References}

You can use references for referring to sections, subsections, figures and so on. As already used in \ref{subsec_fig} you can use this feature to easily refer to other parts of your document.
% use "\label{...}" to define a label and "\ref{...}" to refer to this label.

\section{Heading section two}

This is a short template to get you startet with \LaTeX{}. See 
\href{http://www.latex4ei.de/latex#}{latex4ei}
for further information.

\end{document}